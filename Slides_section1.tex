%+----------------------------------------------------------------------------+
%| SLIDES: Longer presentation on my paper 1805.01696
%| Chapter: Brief introduction to multisymplectic geometry and Homotopy momap
%| Author: Antonio miti
%| Event: Visit in Salerno, March 2019
%+----------------------------------------------------------------------------+

%- HandOut Flag -----------------------------------------------------------------------------------------
\newif\ifHandout
	\Handouttrue  %uncomment for the printable version

%- D0cum3nt ----------------------------------------------------------------------------------------------
\documentclass[beamer,10pt]{standalone}
	%\setbeameroption{show notes}
	




%- Packages ----------------------------------------------------------------------------------------------
%\usepackage{verbatim}
\usepackage[mode=buildnew,subpreambles=true]{standalone}
\usepackage{import}
\usepackage{amsmath, amssymb}
\usepackage{tikz}
%\usetikzlibrary{arrows,shapes,calc}
%\usetikzlibrary{shapes.callouts}
\usepackage{tikz-cd}
\usepackage{hyperref}
\usepackage[english]{babel}
\usepackage{stackengine}

%--Beamer Style-----------------------------------------------------------------------------------------------
\usetheme{toninus}



%--Beamer Style-----------------------------------------------------------------------------------------------
\usetheme{toninus}





%---------------------------------------------------------------------------------------------------------------------------------------------------
%- D0cum3nt ----------------------------------------------------------------------------------------------------------------------------------
\begin{document}
%------------------------------------------------------------------------------------------------


  \subsection{Multisymplectic manifolds}
  \begin{frame}[fragile]{Multisymplectic Manifold} %Fragile -->workaround tikzcd
			\begin{defblock}[$n$-plectic manifold]
			\includestandalone[width=0.95\textwidth]{Pictures/Figure_multisym}	
			\end{defblock}

			\begin{defblock}[(Weekly) Non-degenerate n-form]
				\begin{columns}
					\hfill
					\begin{column}{.4\linewidth}
						\centering{
						The multi-contraction map $\alpha^{(j)}$\par
						is injective in the case $j=1$.
						}
					\end{column}
					\begin{column}{.6\linewidth}
						\[
						\begin{tikzcd}[column sep= small,row sep=0ex]
						    \alpha^{(j)} \colon& \mathfrak{X}^j(M) 	\arrow[r]& 				\Omega^{n+1-j}(M) \\
						  						& \xi_1\wedge\ldots\wedge\xi_j 						\arrow[r, mapsto]& 	\iota_{\xi_j}\cdots\iota_{\xi_1} \omega 					
						\end{tikzcd}	
						\]
					\end{column}
				\end{columns}
			\end{defblock}

			\begin{itemize}
					\item multisymplectic means \emph{going higher} in the rank of $\omega$\pause
					\item 1-plectic $=$ symplectic
			\end{itemize}
			\vspace{1ex}
			\pause
			\begin{block}{Example:}
				Any oriented $(n+1)$-dimensional manifold is $n$-plectic w.r.t. the volume form.
			\end{block}			 


  \end{frame}
  \note[itemize]{
  	\item Notation: in the following, we will suppress $M$ when denoting the spaces of tangent fields $\mathfrak{X}(M)$ and differential forms $\Omega^n(M$)
  	\item Remark: introducing the notation of multi-contraction could seem an overkill at this point. However it will be useful in the following.
  	%Hiny by Ori
  	\item Weak is an $\infty$-dimensional condition on sections. Usually (for instance in the tautological form construction) is used a stronger condition:\\
		The bundle map $\alpha: TM \rightarrow \wedge^n M$, acting on fibers as
		$\alpha_x : v_x \to \iota_{v_x} \omega_x$
		is injective on every fibers.  
		\item the strong implies the weak: a bundle map between bundles on the same base induce a map between sections.
		\item The converse is not true. Consider $\omega = x \text{d}x \wedge \text{d}y$, a presymplectic form that is not degenerate only on a dense domain.
		The map induced on section is injective
		$$\Gamma(T\mathbb{R}^2) \rightarrow \Gamma(T^\ast\mathbb{R}^2) : f\partial_x + g\partial_y \to x \left(f \text{d}y - g \text{d} x\right)$$
		because the r.h.s. vanish only when $f,g$ are zero on $\mathbb{R}^2 \ \text{y-axis}$ but since $f,g$ are coefficient of smooth field, the continuity implies that such functions vanish everywhere.
		
		}
%---------------------------------------------------------------------------------------------------------------------------------------------------
	
%------------------------------------------------------------------------------------------------
\begin{frame}[fragile]{MS geometry and classical field mechanics}
		Consider a smooth manifold $Y$,
		\begin{columns}
			\hfill
			\begin{column}{.5\linewidth}
				\emph{Multicotangent bundle} $\bigwedge = \bigwedge^n T^\ast Y$\\
				is naturally $n$-plectic
			\end{column}
			\begin{column}{.4\linewidth}
				\[
				\begin{tikzcd}
					\Lambda \ar[d,"\pi"'] & T \Lambda \ar[d,"T \pi"] \ar[l] \\
					Y								& T Y \ar[l]
				\end{tikzcd}	
				\]
			\end{column}
		\end{columns}
	\pause
	\begin{defblock}[Tautological $n$-form]
		$\theta \in \Omega^n(\Lambda)$ such that:
		\begin{displaymath}
		\begin{split}
			\left[ \iota_{u_1 \wedge \ldots \wedge u_n} \theta \right]_\eta 
			&= \iota_{(T \pi)_\ast u_1 \wedge \ldots \wedge (T \pi)_\ast u_n} \eta \\
			&= \iota_{u_1 \wedge \ldots \wedge u_n} \pi^\ast \eta 
			\qquad \qquad \forall \eta \in \Lambda \, , \: \forall u_i \in T_\eta \Lambda 		
		\end{split}
		\end{displaymath}
	\end{defblock}
	\vfill
	\begin{columns}
		\begin{column}{.6\linewidth}
			\begin{defblock}[Tautological (multisymplectic) (n+1)-form]
				$$\omega := d \theta$$
			\end{defblock}
		\end{column}
		\begin{column}{.4\linewidth}
		 	\begin{claimblock}$\omega$ is not degenerate.\end{claimblock}	
		\end{column}
	\end{columns}	
	\pause
	\begin{keywordblock}
		\begin{tabular}{|c|c|c|}
			\hline 
			point-particles mechanics & $\rightsquigarrow$ & classical fields mechanics \\
			%(finite discrete DOF) & & (finite dimensional continuous DOF) \\
			\hline 
			symplectic & $\rightsquigarrow$ & multisymplectic \\ 
			\hline 
			Observables (Poisson) algebra & $\rightsquigarrow$ & Observables $L-\infty$ algebra
			 \\ 
			\hline 
			Co-moment map & $\rightsquigarrow$ & Homotopy co-momentum map \\ 
			\hline 
		\end{tabular} 
	\end{keywordblock}

	
\end{frame}
\note[itemize]{
	\item This example is significant from the perspective of geometric classical field theory:
		\begin{displaymath}
			\frac{\text{classical mechanics}}{\text{symplectic geo.}} =
			\frac{\text{classical field mechanics}}{\text{multisymplectic geo.}}
		\end{displaymath}
	\item Multicotangent bundle is the \emph{Higher analogue} of the cotangent bundle.
	(but it is not yet the analogue of a \emph{phase space}.)
\item The multiphase space is the sub-bundle of $n$-forms vanishing when contracted with 2 vertical fields.
  	\item The reason why this sub-bundle has a particular role is that it can be proved to be isomorphic to a suitable dual of the first Jet bundle.
  	\item For further details see Gotay et al. \href{https://arxiv.org/abs/physics/9801019}{arXiv:physics/9801019}. For a pictorial representation of all the structures involved in the geometric mechanics of I order classical field theories see appendix, pag: \ref{frame:Gimmsy}.
}
%------------------------------------------------------------------------------------------------	
	
%------------------------------------------------------------------------------------------------
\begin{frame}{Special classes of smooth objects} 
  	\begin{columns}
		\begin{column}[t]{.42\linewidth}		
			\begin{defblock}[Hamiltonian v.f.]
				$\mathfrak{X}_{ham} =  \left\lbrace X \in  \mathfrak{X} \right\vert \left. \iota_x \omega \textrm{ exact}  \right\rbrace$ 			
			\end{defblock}
			\begin{defblock}[Multisymplectic v.f.]
				$\mathfrak{X}_{ms} =  \left\lbrace X \in  \mathfrak{X} \right\vert \left. \mathcal{L}_X \omega = 0  \right\rbrace$ 	
			\end{defblock}
		\end{column}
		\begin{column}[t]{.58\linewidth}		
			\begin{defblock}[Hamiltonian $(n$-$1)-$forms]
				\begin{displaymath}
					\Omega^{n-1}_{ham} 	:=
					\biggr\{ H \in  \Omega^{n-1} \; \left\vert \; 
					\stackanchor{$\exists X \in \mathfrak{X}_{ham}$}{: $d H = -\iota_X \omega$} \right\} 
			\end{displaymath}
			\end{defblock}		
		\end{column}
  	\end{columns}
  	%
  	\vspace{0.5em}
  	%
  	\onslide<2->{
  	\begin{columns}
		\begin{column}[t]{.5\linewidth}	
			\centering\emph{Global symmetries}
			\begin{defblock}[Multisymplectic (Lie group) action]
				$\Phi: G \circlearrowright (M, \omega)$ \emph{right action} s.t. \\
				$$\hat{\Phi}(g)_\ast \omega = \omega \quad \forall g \in G$$
			\end{defblock}
		\end{column}
		\begin{column}[t]{.5\linewidth}			
			\centering\emph{Infinitesimal symmetries}
			\begin{defblock}[Multisymplectic (Lie algebra) action]
				$V: \mathfrak{g} \rightarrow \mathfrak{X} (M)$ \emph{Lie algebra morphism} s.t. \\
				$$\mathcal{L}_{V_\xi} \omega = 0 \quad \forall \xi \in \mathfrak{g}$$	
			\end{defblock}
		\end{column}
  	\end{columns}
  	}
  	%
  	\onslide<3->{		
	  	\begin{asideblock}[Hierarchy of conserved quantities]%Shades of...
	  		\begin{table}[] % http://tablesgenerator.com/
			\begin{tabular}{lllll}
					& strictly conserved & & & $\mathcal{L}_X \alpha= 0$ \\
				$\alpha \in \Omega^\bullet$ & globally conserved & along $X \in \mathfrak{X}$ & $\Leftrightarrow$ & $\mathcal{L}_X \alpha\in B $ (exact) \\
				  & locally conserved  & & & $\mathcal{L}_X \alpha\in Z $ (closed)                                
			\end{tabular}
			\end{table}
	  	\end{asideblock}
  	}
  	
  \end{frame}
  \note[itemize]{
  	\item Exactly as it happens in symplectic geometry, fixing a smooth form $\omega$ on $M$ yields a criterion for classifying vector fields and differential forms.
  	\\(Pay attention to the sign convention in defining the Hamiltonian vector fields)
  	\item Also, we can naturally select a special class of symmetries (global and infinitesimal) which preserve the fixed multisymplectic form.
  	\item Aside, we can start to see that, in this setting, measurable quantities are not only smooth functions but also differential forms with degree greater then zero.
  	For such objects can be defined weaker notions of conservation along a flow.
  	\item The idea to consider forms of various degree as observables do not fall out of the sky. 
  		For instance in a string there will be two kind of measurable quantities: extensive observable (1-forms), like the density, and intensive observables (0-forms), like the tension. 
 		%\href{https://en.wikipedia.org/wiki/Intensive_and_extensive_properties#Intensive_properties}{(wiki link on this terminology)}
  	\item Starting from this observation we can define the space of all possible observables (see next slide).
  }
%---------------------------------------------------------------------------------------------------------------------------------------------------

%---------------------------------------------------------------------------------------------------------------------------------------------------
  \subsection{Lie $\infty$-algebra of Observables}
  \begin{frame}[fragile,t]{Lie $\infty$-algebra of Observables \emph{(Rogers)}}
  	Consider $(M,\omega)$, $n$-plectic manifold,
	\begin{defblock}[$L-\infty$ Algebra of observables]
		Is a chain-complex\\
		\ifHandout
			\import{Pictures/}{Figure_Observables}	
		\else
			\import{Pictures/}{Frame_Observables}
		\fi
		
		\onslide<2->{with $n$ (skew-symmetric) multibrackets $(2 \leq k \leq n+1)$\\
			\import{Pictures/}{Equation_Multibracket}
		}
		\only<2->{	\footnote{$\varsigma(k) := - (-1)^{\frac{k(k+1)}{2}}$}}

	\end{defblock}

%  \onslide<3->{
		\begin{itemize}
			\item \emph{higher analogue} of the \emph{Poisson algebra structure} associated to a symplectic manifold.
		\end{itemize}
	%}
  \end{frame}
 \note[itemize]{
	 \item What is a $L-\infty$ of observables?\\
		Basically, is a chunk of the de Rham complex of $M$ with inverted grading and an extra structure, namely the multibrackets.
 	\item Take away message: the "space of observables" on a ms. mfd. carry the structure of a $L\-\infty$ algebra.\\
 		( In the symplectic case it reduces to the corresponding Poisson algebra)
 	\item Rogers associated to any n-plectic mfd a $L\-\infty$ algebra.
 	\item Zambon generalized to it pre n-plectic.
 	\item Remark: recognize in the definition of $[\cdot,\ldots,\cdot]_k$ the contraction with hamiltonian fields.
 	\item This definition is a special instance of a more general object  called $L\-\infty$ Algebra. \\
 		An $L-\infty$ algebra is a notion that one obtains from a Lie algebra 
 		requiring that the Jacobi identity is satisfied only up to a higher coherent chain homotopy. (see Appendix)
 }
%------------------------------------------------------------------------------------------------

%------------------------------------------------------------------------------------------------
  \subsection{Homotopy co-momentum map}
  \begin{frame}[fragile,t]{Homotopy co-momentum map \emph{(Callies, Frégier, Rogers, Zambon)}}
  	%
		Consider a multisymplectic action $G \circlearrowright (M, \omega)$,
		%
		\begin{defblock}[Homotopy co-momentum map (HCMM)]				
			Is a sequence of linear maps:
			\begin{displaymath}
				(f)  = \big\lbrace f_k: \; \Lambda^k{\mathfrak g} \to L_{k-1} \subseteq \Omega^{n-k} 
				\;\big\vert\; 0\leq k \leq n+1  \big\rbrace
			\end{displaymath}
			%
			\includestandalone[width=0.95\textwidth]{Pictures/Frame_HCMM}			
			\emph{such that:}
			\begin{itemize}
				\item $f_0 = 0 $, $f_{n+1} = 0$ %(we have tacitly set $\Lambda^{-1}(M) = 0$)
				\item<2-> $-f_{k-1} (\partial p) = d f_k (p) + \varsigma(k) \alpha^{(k)} \quad \forall (k=1,\dots n+1), \; \forall p \in \Lambda^k(\mathfrak{g})$
			\end{itemize}
		\end{defblock}
		\begin{itemize}
			\item \emph{Higher analogue} of the ordinary co-moment map $f\colon \mathfrak{g}\rightarrow C^\infty(M)$.
		\end{itemize}
  \end{frame}
  \note[itemize]{
		\item Notice that a HCMM pertains to an "infinitesimal" action of ${\mathfrak g}$ on $M$ 
			with ${\mathfrak g}$ being the Lie algebra of a generic Lie group $G$, 
			acting on $M$ by $\omega$-preserving vector fields.
		\item (Not.) $ p = \xi_1 \wedge \xi_2 \wedge \dots \wedge \xi_k$, 
			then $v_p = v_1 \wedge v_2 \wedge \dots \wedge v_k$ 
			where $v_i \equiv v_{\xi_i}$ are the fundamental vector fields associated to the action $G \circlearrowright M$.
	%	\item (Notation) $\iota(v_p) \omega = \iota(v_k)\dots\iota(v_1) \omega$
	%	\item $\varsigma(k) := - (-1)^{\frac{k(k+1)}{2}}$ 
		\item (Recall) $\alpha^{(k)}:= \iota(v_p) \omega = \iota(v_k)\dots\iota(v_1) \omega$
		\item $\partial \equiv \partial_k:  \Lambda^{k} {\mathfrak g} \to \Lambda^{k-1} {\mathfrak g}$  is the usual Eilenberg-Chevalley complex boundary operator (see appendix, pag: \ref{frame:CE-complex});
		\item The definition tells us that the {\it closed} forms
			$$\mu_k := f_{k-1} (\partial p) +  \varsigma(k) \iota(v_p) \omega 	$$
			must actually be {\it exact}, with potential $-f_k(p)$.  	
		\item The last equation tells us that an HCMM is not a chain complex morphism but is rather a chain complex homotopy between 0 and the multicontraction $\alpha$ (is a chain map by lemma 2.18 \cite{Ryvkin2016}).
		\item An HCMM is a L-$\infty$ morphism 
		$(f):\mathfrak(g)\rightarrow L(M)$ s.t. 
		$d f_1(\xi) = -\iota_{v_\xi} \omega$.\\
		\footnotesize{(Compare with the definition of ordinary co-moment map: \\Lie algebra morphism 
		$J:\mathfrak(g)\rightarrow C^\infty(M)$ s.t. $ d J(\xi) =  -\iota_{v_\xi} \omega$)}
		
  }
%------------------------------------------------------------------------------------------------

%------------------------------------------------------------------------------------------------
\begin{frame}[fragile]{ A Cohomological existence condition (C.F.R.Z.)} 
%
	Consider a multisymplectic action $G \circlearrowright (M, \omega)$, fix $x\in M$
				\[
				\begin{tikzcd}[column sep= small,row sep=0ex]
				    C_x :& \wedge^{n+1} \mathfrak{g}	\arrow[r]& 
				    \mathbb{R} \\
				    	& p= x_1\wedge\ldots\wedge x_{n+1} \arrow[r, mapsto]&
				    	\iota(v_p) \omega \rvert_x
				\end{tikzcd}	
				\]
	is a (Chevalley-Eilenberg) co-cycle, \quad i.e $\quad\delta_{CE} C_x = C_x \circ \partial = 0$
	%
	\onslide<2->{
	\begin{propblock}[
			\begin{tabular}{p{2cm} p{0.5cm} p{6.5cm}}
				$M$ connected &  $\Rightarrow$ & the cohomology class 
				$\lbrack C_x \rbrack_{\mathfrak{g}}$ is independent of the choice of point $x$
			\end{tabular}]
		\underline{Sketch:} One finds that: $C_{x^{\prime}} - C_x =  \delta_{CE} (b)$, where
			\begin{displaymath}
				b (\xi_1\wedge \xi_2 \wedge\dots \wedge \xi_{n}) := -\varsigma (n+1) \int_{\gamma} \iota (v_1\wedge v_2 \wedge\dots \wedge v_{n}) \omega			
			\end{displaymath}
			and $\gamma$ is a path connecting $x$ to $x^{\prime}$. 
	\end{propblock}
	}
	%
	\onslide<3->{
	\begin{propblock}[
			\begin{tabular}{p{2cm} p{1cm} p{6.5cm}}
				$\exists$ $\mathfrak{g}$-HCMM & $\Rightarrow$& $\lbrack C_x \rbrack_{\mathfrak{g}} = 0$\\
				&($\Leftarrow$& if   $H^i(M)=0$ for $1\leq i\leq n-1$)
			\end{tabular}
			]
		See \href{https://arxiv.org/abs/1304.2051}{Callies et al}, Section 9.1.
	\end{propblock}
	}
\end{frame}
\note[itemize]{
	\item Beware to the sloppy notation in the proposition, $v_i = v_{\xi_i}$.
	\item A more refined statement can be found in \cite{Fregier2015}\\
		HCMM are in a one to one correspondence with primitives of the cocycle
		$\tilde{\omega} = \sum (-)^{k-1} \omega_k $ with
		$$
				\omega_k : \wedge^k\mathfrak{g} \ni x_1 \wedge \ldots \wedge x_k \mapsto
				\iota(v_1\wedge\ldots\wedge v_k) \omega \in \Omega^{n-k}
		$$
		in the chain complex
		$$
			\mathcal{C} = \left(\wedge^{\geq 1} \mathfrak{g}^\ast \otimes \Omega(M), d_{\text{tot}} = 
			d_{\mathfrak{g}}\otimes 1 + 1 \otimes d_{\text{dR}} \right)
		$$
 }
%-------------------------------------------------------------------------------------------------------------------------------------



\end{document}





