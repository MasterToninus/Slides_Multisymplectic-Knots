%- HandOut Flag -----------------------------------------------------------------------------------------
\newif\ifHandout

%- D0cum3nt ----------------------------------------------------------------------------------------------
\documentclass[beamer,10pt]{standalone}



%- Packages ----------------------------------------------------------------------------------------------
\usepackage{verbatim}
\usepackage{appendixnumberbeamer}

\usepackage{amsmath, amssymb}
\usepackage{tikz}

\usepackage{tikz-cd}
\usepackage{graphicx, animate}
\usepackage{hyperref}
\usepackage[english]{babel}
\usepackage{csquotes}
\usepackage[mode=buildnew,subpreambles=true]{standalone}
\usepackage{stackengine}


\providecommand{\blank}{\text{\textvisiblespace}}

%--Beamer Style-----------------------------------------------------------------------------------------------
\usetheme{toninus}

%--WORKAROUND------------------------------------------------------------------------------------------
%Credit: https://tex.stackexchange.com/questions/147899/path-problem-with-included-file-inside-of-a-standalone-file
\providecommand{\includestandalonewithpath}[3][]{%
  \begingroup%
  \providecommand{\datapath}{#2}%
  \includestandalone[#1]{\datapath/#3}%
  \endgroup}




%---------------------------------------------------------------------------------------------------------------------------------------------------
%- D0cum3nt ----------------------------------------------------------------------------------------------------------------------------------
\begin{document}

%------------------------------------------------------------------------------------------------
% APPENDIX
%------------------------------------------------------------------------------------------------
\appendix
\section*{VECCHI SPUNTI}
\sectionpage
%------------------------------------------------------------------------------------------------

%------------------------------------------------------------------------------------------------
\begin{frame}{parte covariant dell'articolo}
As a byproduct, we shall exhibit a covariant phase space interpretation of Brylinski's manifold of mildly singular links upon resorting the Euler equation for perfect fluids.
 Pagina 22 slides MS
\end{frame}
%------------------------------------------------------------------------------------------------

%-------------------------------------------------------------------------------------------------------------------------------------
  \begin{frame}[fragile]{Capire/Motivare algebra $L-\infty$ degli osservabili}
	Perchè ho una lista di osservabili dalle 0-forme alle n-1 forme?\\
	Gimmsy ci dice:
	\begin{itemize}
		\item ad un sistema fisico con con set dei gradi di libertà continuo di dimensione k+1
		\item si associa un fibrato $E \to M$ con varietà base di demnsione k+1 ("spazio"+"tempo")
		\item a cui si associa una varietà k+2-plettica
		\item gli osservabili sono dalle k-forme in giù. Bene è tutto ciò che si può pullbackare 
			sui "piano di contemporaneità" $\Sigma$ senza che svanisca.
		\item ovvero gadget che si possono valutare su punti, direzioni e opportune sottovarietà di $\Sigma$
	\end{itemize}
  	E.g. il Filamento:
	%
		\[ 
	\begin{tikzcd}[column sep= small,row sep=0ex]
		E \ar[r,equal] & \mathbb{R}\times I \times \mathbb{R}^3 \\
		M \ar[r,equal] & \mathbb{R}\times I
	\end{tikzcd} 
	\]	
  \end{frame}
  \note{
  		Motivations:\\
		is known that filament (knot) system enjoys important symmetries. Namely is invariant under  orientation preserving transformations.
		Therefore is natural to look at the momenta associated. MS geometry and HCMM gives the language to formalize this concept.
		\\
		Our construction is a special case of Thm 9.6 CFRZ. Note that in cooking our $\phi$ we only relied on Riemannian geometric feature. So, a more abstract extension is in order.
		Note also the importance of the vanishing at the point of all the fields.
		\\
	}
%------------------------------------------------------------------------------------------------



\end{document}
