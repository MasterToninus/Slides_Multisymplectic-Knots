%+------------------------------------------------------------------------+
%| ABSTRACT: Longer presentation on my paper 1805.01696
%| Author: Antonio miti
%| Event: Visit in Salerno, March 2019
%| 
%+------------------------------------------------------------------------+

\documentclass[11pt,a4paper,twoside]{article}

\usepackage{fancyhdr}
%\usepackage[T1]{fontenc}
\usepackage{amsfonts}

%-----------  SIZE  ------------%
\parskip=0.5ex
\oddsidemargin= 0.35cm
\evensidemargin= 0.35cm

\parindent=1.5em
\textheight=23.0cm
\textwidth=15.5cm


\begin{document}

\title{Multisymplectic aspects of link invariants}
\author{Antonio Michele Miti\\ UCSC Brescia \& KU Leuven}

\date{March 18th, 2019}
\maketitle


\begin{abstract}

The present talk is a survey of part of recent joint work with Mauro Spera\cite{firsticle} (arXiv: 1805.01696), in which we investigated some connections between multisymplectic geometry and knot theory.
\\
A connection between these two topics can be established via mechanics of ideal fluids.
The key idea is to regard the group of orientation-preserving diffeomorphism of the Euclidean space (corresponding to spatial configurations of an ideal incompressible fluid permeating the whole space) as a multisymplectic action on $\mathbb{R}^3$ with the standard volume form seen as a 2-plectic form.
\\
As a first result, we can explicitly construct a homotopy co-momentum map (à la Callies,  Fregier, Rogers and Zambon \cite{CFRZ}) associated to this multisymplectic action showing that it correctly transgresses to the standard hydrodynamical co-momentum map defined by Arnol’d, Marsden and Weinstein and others.
\\
The transition to knots occurs when one considers vortex filaments in hydrodynamics.
It is possible to associate to these peculiar configurations of the fluid suitable conserved quantities, as defined by Ryvkin, Wurzbacher and Zambon\cite{RWZ}. 
These quantities are directly related to the Gauss linking number of the link supporting the vorticity.
\\
Time permitting, we shall discuss a reinterpretation of the (Massey) higher order linking numbers in terms of conserved quantities within the \cite{RWZ} multisymplectic framework, giving rise to knot theoretic analogues  of first integrals in involution.
 
\end{abstract}




\begin{thebibliography}{9}
	\bibitem{firsticle}
		Antonio Michele Miti and Mauro Spera.
		\newblock On some (multi)symplectic aspects of link invariants, 2018;
		\newblock arXiv:1805.01696.
	%
	\bibitem{CFRZ}
		Martin Callies, Yael Fregier, Christopher L. Rogers and Marco Zambon.
		\newblock Homotopy moment maps, 2013;
		\newblock arXiv:1304.2051.
	%
	\bibitem{RWZ}
		Leonid Ryvkin, Tilmann Wurzbacher and Marco Zambon.
		\newblock Conserved quantities on multisymplectic manifolds, 2016;
		\newblock arXiv:1610.05592.

\end{thebibliography}

\end{document}
