%+------------------------------------------------------------------------+
%| Frame: Anatomy of Generalized Poisson sign
%| Credit: https://tex.stackexchange.com/questions/15735/adding-arrows-to-each-term-of-an-equation
%| Author: Antonio miti
%+------------------------------------------------------------------------+
\providecommand{\datapath}{.}

\documentclass[beamer,10pt]{standalone}
\usepackage{mathtools}
\usepackage{amsfonts}
\usepackage{amsmath}
\usepackage{tikz}
\usetikzlibrary{arrows}
\usepackage{graphicx, animate}


\usepackage[most]{tcolorbox}
\newtcolorbox{REdefblock}[1][]{
	colback=white,
	colbacktitle=white,
	coltitle=red!70!black,
	colframe=red!70!black,
	%boxrule=1pt,
	titlerule=0pt,
	%arc=15pt,
	left skip=0em, right skip=0em,
	%enlarge left by=-1em,width=\linewidth+1.5em, %box horizontal size
	left=0em,right=0em,top=0em,bottom=0em, %text spacing
	toptitle=-0.2em,bottomtitle=-0.2em, %title spacing
	title={Def: \strut#1}
}


\tikzstyle{every picture}+=[remember picture]
\everymath{\displaystyle}

\makeatletter

% Designate a term in a math environment to point to
% Syntax: \mathterm[node label]{some math}
\newcommand\target[2][]{%
   \tikz [baseline] { \node [draw=black,fill=white,rectangle,anchor=base] (#1) {#2}; }}
\newcommand\mathtarget[2][]{%
   \tikz [baseline] { \node [draw=black,fill=white,rectangle,anchor=base] (#1) {$#2$}; }}
\newcommand\source[2][]{%
   \tikz [baseline] { \node [inner sep=0pt,anchor=base] (#1) {#2}; }}
\newcommand\mathsource[2][]{%
   \tikz [baseline] { \node [inner sep=0pt,anchor=base] (#1) {$#2$}; }}
   
% A command to draw an arrow from source to target
% Default color=black, default arrow head=stealth
% Syntax: \indicate[color]{source}{target}[path options]
\newcommand\indicate[3][black]{%
   \@ifnextchar[{\@indicateopts{#1}{#2}{#3}}{\@indicatenoopts{#1}{#2}{#3}}}
\def\@indicatenoopts#1#2#3{%
   {\color{#1} \tikz[overlay] \path[line width=1pt,draw=#1,-stealth] (#3) edge (#2);}}
\def\@indicateopts#1#2#3[#4]{%
   {\color{#1} \tikz[overlay] \path[line width=1pt,draw=#1,-stealth] (#3) [#4] edge (#2);}}

\makeatother


\begin{document}
\begin{standaloneframe}[fragile]{$L-\infty$ Algebras - about the "sign" term} %TODO 
	\begin{columns}
	    \begin{column}{.5\linewidth}
	    		\source[s1]{
			\begin{tcolorbox}[colback=white]	
				q
		  	\end{tcolorbox}  		
	    		}
		\end{column}
		%
	    \begin{column}{.5\linewidth}
	    		\source[s4]{
			\begin{tcolorbox}[colback=white]	
				q
		  	\end{tcolorbox} 		
	    		}
		\end{column}
	\end{columns}    	 

	\begin{center}
		\begin{displaymath}
			\Big[\text{"sign"} \Big]\, =	\mathtarget[t1]{(-)^{i(j+1)}}
			\mathtarget[t2]{\epsilon(\mathtarget[t3]{\sigma};x_1,\ldots,x_k)} \mathtarget[t4]{\text{sign}(\sigma)}
		\end{displaymath}	
		\vspace{5ex}
	\end{center}
	
	\begin{columns}
	    \begin{column}{.5\linewidth}
	    		\source[s3]{
			\begin{REdefblock}[Unshuffles]	    		
				\begin{columns}
				%
			    \begin{column}{.6\linewidth}
			    \vspace{-5ex}
			    	\begin{displaymath}
			    	\begin{split}
						&ush(p,q) = \\
						&\big\lbrace	\sigma \in S_{p+q} \,\big\vert\substack{\sigma(i) < \sigma(i+1)\\ \forall i \neq p}\big\rbrace  		
				\end{split}
			    	\end{displaymath}
			    \end{column}  
				%		
			    \begin{column}{.4\linewidth}
					%http://theknaveofclovers.tumblr.com/post/41608490725/card-shuffle
					%\animategraphics[autoplay,loop,width=\linewidth]{5}{\datapath/unshuffle/unshuffle-}{0}{16}
			    \end{column}  
				%
				\end{columns}
		  	\end{REdefblock}    		
	    		}

		\end{column}
		%
	    \begin{column}{.5\linewidth}
	    		\source[s2]{
				\begin{REdefblock}[Kosuzl sign]
					signature of the restriction of $\sigma$ to the subset of indices 
					$i$ such that $v_i$ has odd degree
				\end{REdefblock}
			}
		\end{column}
	\end{columns}    	    	
	
	%Drawing overlay arrows
	% \tikz[overlay] \path[line width=1pt,draw=red,-stealth] (s3) edge (t3);
	\indicate[red]{t1}{s1}[out=-90,in=95]
	\indicate[red]{t4}{s4}[out=-90,in=95]
	\indicate[red]{t2}{s2}[out=90,in=-95]
	\indicate[red]{t3}{s3}[out=90,in=-95]
		
\end{standaloneframe}
\note[itemize]{
	\item \href{Manetti - Notes}{http://www1.mat.uniroma1.it/people/manetti/DT2011/Linfinitoalgebre.pdf}
   \item Koszul sign keep track of the extra sign that comes out when permuting element in odd degree
   \item see Leonid thesis for a great description of all the terms! \url{https://www.springer.com/gp/book/9783658123895}
}
%------------------------------------------------------------------------------------------------





\end{document}

	