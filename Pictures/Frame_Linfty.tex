%+------------------------------------------------------------------------+
%| Frame: Lie infinity algebra definition
%| Author: Antonio miti
%+------------------------------------------------------------------------+

\documentclass[beamer]{standalone}
\usepackage{tikz}
\usetikzlibrary{arrows,shapes}
\usepackage{amsmath, amssymb}




\begin{document}
\begin{standaloneframe}[fragile]
				%
				\tikzstyle{every picture}+=[remember picture]
				\everymath{\displaystyle}
				\tikzstyle{na} = [baseline=-.5ex]
				%
				\begin{columns}
			    \begin{column}{.3\linewidth}
			\begin{displaymath}
				 \big(
					 \tikz[baseline]{
					            \node[fill=blue!20,anchor=base] (t1)
					            {$ L$};
					        } 
						,
						 \tikz[baseline]{
					            \node[fill=blue!20,anchor=base] (t2)
					            {$ \lbrace l_i \rbrace_{i \in \mathbb{N}} $};
						}
				\big)
			\end{displaymath}	
			    \end{column}
			    %
			    \begin{column}{.6\linewidth}		
		\begin{itemize}
		    \item[] \tikz[na] \node[coordinate,fill=blue!20,draw,circle] (n1) {};		    
		   		 Graded vector space: $L = \bigoplus_{i\in\mathbb{Z}} L_i$
		    \item[] \tikz[na]\node [coordinate,fill=blue!20,draw,circle] (n2) {};	    
Family of maps (\emph{multi-brackets}) 
			 $$l_k=[\cdot,\ldots,\cdot]_k:\wedge^k L\to L \quad,\qquad\deg(l_k)=2-k$$
	\end{itemize}	
			    \end{column}
			    \end{columns}
				%
				\begin{tikzpicture}[overlay]
				        \path[->] (n1) edge [bend right] (t1);
				        \path[->] (n2) edge [bend left] (t2);
				\end{tikzpicture}
				%
$\displaystyle
			\sum_{i+j=k+1}(-)^{i(j+1)}\sum_{\sigma\in\text{ush}(i,k-i)}
			\Big[ \epsilon(\sigma;x_1,\ldots,x_k) \text{sign}(\sigma) \Big]\,
			\Big[ 	l_j\left(l_i\left(x_{\sigma_1},\ldots,x_{\sigma_i}\right),x_{\sigma_{i+1}},\ldots,x_{\sigma_k}\right)\Big]
			= 0
$
\end{standaloneframe}
\end{document}
